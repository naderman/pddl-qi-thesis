% Use UTF-8 encoding!!
% (c) Stefan Ulbrich, 2012

\documentclass[english,ngerman]{KITreprt}


%% -------------------------------
%% |  Information for PDF file   |
%% -------------------------------
\hypersetup{
 pdfauthor={Nils Adermann},
 pdftitle={Diplomarbeit Nils Adermann},
 pdfsubject={Not set},
 pdfkeywords={Not set}
}


%% --------------------------------
%% Obligatory Parameters:
%% --------------------------------

\renewcommand{\myname}{Nils Adermann}
\renewcommand{\mythesis}{\diplomarbeit} %\mastersthesis, \bachelorsthesis, \protocol, \studienarbeit, \diplomarbeit
\renewcommand{\myshorttitle}{}
\renewcommand{\mytitle}{Communicating, Planning, and Executing Object-Action Complexes}

\renewcommand{\timestart}{October 1\textsuperscript{st}, 2013}
\renewcommand{\timeend}{March 31\textsuperscript{st}, 2013}
\newcommand{\advisor}{Prof. Dr.-Ing. Tamim Asfour}
%\newcommand{\advisortwo}{Nur wenn n\"otig}

\graphicspath{{./figures/}}


\begin{document}


\selectlanguage{english}


\maketitle


\tableofcontents
\chapter{Motivation}
The majority of robots in use today solve only one or few specific problems in
easily predictable constraint environments. Rigid requirements on predictability
may suit industrial robots working in assembly lines but they pose a problem for
general purpose service robots. A service robot collaborating with humans in
environments made for humans like a household will encounter a much wider variety
of objects and activities. A service robot operating in these environments will
without doubt have to cope with unexpected situations.

Artificial intelligence has been a research subject for more than 50 years and
many results have been found using simulation or simple robots. But only recent
advances in hardware and mechanics have made it feasible to build general
purpose robots like humanoid robots. The open environments these operate in and
inaccuracies of their sensors pose additional problems for artificial
intelligence. Decisions need to be made relying on uncertain data, and unexpected
situations need to be analyzed and understood to react appropriately.

This thesis intends to demonstrate that a robot agent can adapt and plan its
activities in accordance with dialog held with a human agent or another robot
agent using the concept of Object-Action Complexes to communicate its
perceptions and plans. It will further show that Object-Action Complexes are a
suitable mechanism for learning how to apply existing knowledge to new
situations using information provided through dialog with another agent.

The robot is fitted with a dialog system to allow a user to provide the robot
with information about its environment and its capabilities. The user can
instruct the robot to perform complex tasks. The robot uses a symbolic task
planning system to arrive at a series of actions to be executed to achieve the
goal selected by the user. The task planner is also used to plan further dialog
contributing information required to execute steps of the plan in order to
achieve the goal.

The robot's knowledge of its own capabilities and objects in its environment is
represented in Object-Action Complexes. Object-Action Complexes are symbolic
representations of sensorimotor experience grounded in real-world experience of
the robot. They are designed to formalize adaptive and predictive behaviors at
all levels of a cognitive processing hierarchy \cite{Geib2011}. Thus the robot can
communicate with other agents about abstract Object-Action Complexes which are
still grounded in actual sensorimotor experience. As Object-Action Complexes
additionally provide sufficient formalization to use them in high-level symbolic
task planning, they allow the robot to execute complex tasks involving
communication.

Throughout this thesis we assume that the robot has already gained various
sensorimotor experiences recorded in Object-Action Complexes. The goal is now to
find suitable mechanisms to generate abstractions of these Object-Action
Complexes which aid communication. These abstractions categorize various
Object-Action Complexes into more generic groups suitable for translation into
spoken language. These categories enable the robot to predict that a particular
new Object-Action Complex might be executable if it can place the description
of the Object-Action Complex in the same category as existing Object-Action
Complexes. As a result the robot can learn new Object-Action Complexes through
spoken language which it can still actually execute by transfering sensorimotor
experiences from similar Object-Action Complexes. Experimentation then allows
the robot to refine these new Object-Action Complexes according to specifics
not directly clear from just the category of Object-Action Complexes it is a
part of.

To evaluate the proposed mechanism for learning categories of Object-Action
Complexes the execution of Object-Action Complexes is combined with
dialog in a scenario that involves tightly coupled interaction between two
agents. The agents need to cooperate in order to achieve a complex goal
together. They can only reach their goal by communicating sensorimotor
information including perceived force exerted by the other agent. The symbolic
task planner is responsible for deciding which Object-Action Complexes to
execute and which information needs to be relayed to the other participating
agent in the form of abstract Object-Action Complexes.

Goals:
\begin{itemize}
    \item Execution of OACs chosen through learned categories of OACs
    \item Application of OACs to unknown (on semantic level) objects assumed to be
        applicable because of spoken dialog and OAC categorization
    \item Acquiring knowledge about physically executable OACs from dialog
        \begin{itemize}
            \item Example for grounding of spoken communication about plans \& OACs
                in physical embodiment
        \end{itemize}
\end{itemize}

\chapter{State of the Art}
\begin{itemize}
    \item Formal definition of OACs and previous uses of OACs
        \begin{itemize}
            \item Relationship between OACs and language
            \item Grounding \& Executing OACs
            \item Learning OACs
        \end{itemize}
    \item Dialog systems
        \begin{itemize}
            \item Speech recognition
                \begin{itemize}
                    \item one4all (Waibel)
                \end{itemize}
            \item Natural Language Generation
            \item Maintaining conversational state
        \end{itemize}
    \item Planning in AI
        \begin{itemize}
            \item PDDL
            \item PKS
            \item Execution of symbolic plans on real robots (PACO-PLUS/Studienarbeit)
        \end{itemize}
    \item Planning dialog with symbolic planners
        \begin{itemize}
            \item Combining Dialog Planning with Task Planning
            \item Dialog Planning with PKS
        \end{itemize}
\end{itemize}

Questions:
Where do predicates come from?
How are predicates and continuous sensor data linked
  - discretization processes from attributes / perception
How can a task specific domain be chosen?
How is the level of abstraction chose?
Are there even clear hierarchies of OACs?
Can one plan on a higher level without considering a lower level?
Excluding spatial information? Motion planning -> task planning
-> but even humans don't consider what they will do 20 steps ahead when performing an action, unless they have learned from experience to consider that particular other action far in the distance -> how does this relate to working memory \& attention?

Memory Models: LTM/Working Memory
How to represent knowledge of actions \& objects in memory
How to use memory in the context of planning
How are planning domains related to LTM?
How is plan monitoring related to working memory validation?

\chapter{A Cognitive Architecture for Object-Action Complexes}

\section{MemoryX}
\begin{itemize}
    \item Overview
    \item Working Memory Segments
        \begin{itemize}
            \item Object Instances
            \item Relations
            \item Active OAC
        \end{itemize}
    \item Long Term Memory Segments
        \begin{itemize}
            \item Object Instance Classes
            \item Relation Classes (?)
            \item OACs
            \item Episodes
        \end{itemize}
\end{itemize}

\section{The Central Executive Agent}
\begin{itemize}
    \item Processes
        \begin{itemize}
            \item Perception
            \item Execution / Simulation
            \item Task Selection
            \item Planning
            \item Learning
        \end{itemize}
\end{itemize}

\section{Planning}
\begin{itemize}
    \item PDDL / PKS
    \item Interface
\end{itemize}

\section{Evaluation and Discussion}


\chapter{Integration of a dialog system with Working Memory}

\section{Evaluation and Discussion}

\chapter{Planning with Dialog}

\section{Evaluation and Discussion}
\begin{itemize}
    \item Examples (show that it works at all)
        \begin{itemize}
            \item ARMAR wipes a table with cloth and learns about sponges
            \item ARMAR and human carry a large board together, ARMAR gives instructions
                (through the door, left right, duck)
        \end{itemize}
    \item ???? categorization of OACs through generated dialog (Peter) taking place in simulation
\end{itemize}

\begin{itemize}
    \item Integrating new version of PKS with ARMAR based on SPOAC
    \item Working Memory system by Alexey
        \begin{itemize}
            \item Compare to SPOAC
        \end{itemize}
    \item Changes to SPOAC to be able to detect new "unknown" objects
    \item Designing PKS domains for examples
    \item Extending SPOAC perceptors with all sensor data required for the examples
    \item SPOAC perceptor for force feedback
    \item Defining an encoding for information about tightly coupled interaction in OACs
    \item Integration with speech recognition system for demos
    \item Creating a new main scenario controller which deals with language input and
        starts planning only when appropriate
\end{itemize}

memoryx introduction - alexey reference
framework for segmentable memory
new segments: oacsegment, activeoacsegment, episode segment

episodic memory implementation

PDDL, pddl parser / AST / PKS transformation / execution - some papers on building parsers/compilers/vms

gui: step by step simulation (like debugger)

simulation allows specifying particular failures by providing override PDDL effect for particular OACs in particular situations

\chapter{Conclusion}


\bibliographystyle{abbrvnat} %-English

%\bibliographystyle{dinat} %-Deutsch

\bibliography{thesis}

\end{document}
